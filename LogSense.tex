\documentclass[pdftex,11pt,a4paper]{book}
%\documentclass[dvips,11pt,a4paper,oneside]{book}

\usepackage {ifpdf}
\usepackage [american,german] {babel}
\usepackage[numbers,comma,square]{natbib}
\usepackage[utf8]{inputenc}
\usepackage{mya4wide}
\usepackage{fancyhdr}
\pagestyle{fancy}
\fancyhf{}
\rhead{LogSense}
\lhead{HTL Perg}
\renewcommand{\headrulewidth}{0pt}

\fancypagestyle{plain}{
  \fancyhf{}
  \rhead{LogSense}
  \lhead{HTL Perg}
  \fancyfoot[R]{\thepage} % Page number format on the right side
  \fancyfoot[L]{Borbely, Ettlinger, Jilek, Stadlbauer} % Names on the left side
  \renewcommand{\headrulewidth}{0pt}
}

\fancyfoot{}
\fancyfoot[R]{\thepage} % Page number format on the right side
\fancyfoot[L]{Borbely, Ettlinger, Jilek, Stadlbauer} % Names on the left side

\begin{document}

\maketitle

\tableofcontents

\chapter*{Eidesstattliche Erklärung}
This chapter introduces the background and motivation behind the software engineering project.

\chapter*{Gender-Erklärung}
Define the scope of the project, including any limitations and constraints.

\chapter*{Danksagung}
Summarize relevant literature and related work in the field of software engineering.

\chapter*{Impressum}
Clearly state the problem the project aims to address.

\chapter*{Kurzfassung}
Define the specific objectives and goals of the software engineering project.
\textbf{Problemstellung} 

\chapter*{Abstract}
Define the specific objectives and goals of the software engineering project.

\chapter*{Inhaltsverzeichnis}

\chapter{Einleitung}
\section{Ausgangssituation}
\section{Problemstellung}
\section{Zielsetzung}
\section{Projektinhalt}
\subsection{Anforderungen}
\subsection{Projektteam}
\subsubsection{Aufgabenverteilung}
\subsection{Auftraggeber}

\chapter{Theoretische und fachpraktische Grundlagen und Methoden}
\section{Visualiserung}
\subsection{Angular}
\subsection{Charst JS}
\subsection{BootStrap}
\subsection{Angular Material}
\section{Datenverarbeitung}
\subsection{Python}
\subsection{Pandas}
\text{Pandas ist eine Python-Bibliothek zur Datenverarbeitung und -analyse. Sie stellt verschiedene Datenstrukturen und Funktionen für das Manipulieren von tabellarischen und strukturierten Daten zur Verfügung. Zu den wichtigsten Strukturen gehören DataFrames und Series. Series sind eindimensionale, array-ähnliche Objekte welche wie eine einzelne Spalte in einer Tabelle sind. Bei DataFrames handelt es sich um zweidimensionale Tabellen mit Zeilen und Spalten. Einzelne Spalten von einem DataFrame können dabei verschiedene Datentypen aufweisen. Zu den wichtigsten Funktionen gehören Aggregationen, Gruppierungen und Statistische Operationen. Bei dieser Diplomarbeit wird Pandas bei der Datenverarbeitung eingesetzt, insbesondere in Kombination mit Scikit-Learn, Numpy und Ruptures zur Identifikation von Anomalien, Change-Points und zur Bereining und Zusammenführung von Datensätzen.}
\subsection{Numpy}
\subsection{Scikit-Learn}
\subsection{Ruptures}
\subsection{Algorithmen}
\section{Datenerfassung}
\subsection{Agent}
\subsection{Java}
\subsection{Oshi}
\section{Datenhaltung}
\subsection{Postgres}
\subsection{Timescale DB}
\section{Schnittstellen}
\subsection{Fast API}
\subsection{REST}
\section{Server}
\subsection{Ubuntu}
\subsection{nginx}

\section{Entwicklungssysteme}
\subsection{PyCharm}
\subsection{IntelliJ}
\subsection{Webstorm}
\subsection{Datagrip}
\section{Sonstige Software}
\subsection{GitLab}
\subsection{Adobe XD}
\subsection{LucidChart}

\chapter{Planung und Realisierung}
\section{Funktionalität (UC, UC-Beschreibung)}
\section{Architektur}
\section{Informationsfluss}
\section{Datenmodell}

\chapter{Implementierung (Programmierung und QS/Test)}
\section{Visualiserung}
\section{Datenverarbeitung}
\subsection{Zusammenführung}
\subsection{Anomalienerkennung}
\subsubsection{Einleitung}
\text{
Ein Ziel der Anwendung ist es, bei den PC- und Anwedungsdaten Extremwerte und Abweichungen zu identifzieren und diese für den Benutzer zu markieren. Anomalienerkennung bei dieser Diplomarbeit hat den Sinn, ungewöhnliches Verhalten von Applikationen den Benutzer aufzuzeigen und anhand dessen ihn bei der Erkennung von Problemen beim Rechner behilflich zu sein. In Kombination mit Justifications bekommt dieser Hinweise darauf, weswegen diese Extremwerte zustandekommen. Auch allgemein spielt die Erkennung von Extremwerten bzw. Anomalien eine große Rolle im Datenverarbeitungsbereich, hauptsächlich aufgrund der Bereinigung von Datensätzen.
}
\subsubsection{Implementierungsmöglichkeiten}
\text{
Für das Erkennen von Extremwerten bieten sich eine Vielzahl von Möglichkeiten an, dazu zählen etwa das Aufteilen der Datensätze in Quantile, das Verwenden von Machine-Learning Algorithmen wie Isolation Forest & K-means oder das Anwenden von Stastisischen Methoden in Verbindung mit eigenen Algorithmen. 
}
\subsubsection{Händische Implementierung}
\text{
Beim der ersten Implementierung sind Extremwerte anhand von bestimmten, prozentuellen Abweichungen vom gleitenden Durschnitt erkannt worden. Zunächst ist dafür der gleitende Durschnitt von Datensätzen berechnet worden und dann die prozentuelle Änderung von einem Datenpunkt zum gleitenden Durschnitt. Sollte dieser Wert eine definierte Grenze, beispielsweise 30 Prozent überschreiten, wird der Datenpunkt als Extremwert identifiziert. Dies ist zwar ausreichend gewesen, um die meisten Extremwerte zu erkennen, jedoch ist diese Vorgehensweise nicht geeignet für Daten mit hoher Varianz.
}
\subsubsection{Anomalienerkennung mit Isolation Forest}
\text{
Zur Anomalienerkennung ist der Machine-Learning-Algorithmus Isolation Forest zur Verwendung gekommen. Dieser basiert auf das Konzept der Isolation, also der Tatsache, dass Anomalien weniger isoliert sind als herkömmliche Datenpunkte. Isolation Forest ist eine logische Wahl für unseren Anwedungszweck gewesen, da es keinerlei Vorverarbeitung unserer Daten benötigt, Daten mit hoher Varianz problemlos funktionieren und im Vergleich zu einigen Alternativen keine Annahmen über die Verteilung der Daten benötigt. Bei der Implementierung sind Probleme bezüglich Underfitting aufgretreten falls nur eine geringe Menge an Datensätzen zum Trainieren des Algorithmus verfügbar gewesen ist. Dies ist der Fall gewesen, wenn der Benutzer die Datenanalyse in einem Zeitrahmen angeforderte wo nur eine geringe Menge an Datensätzen verfügbar war. Als Lösung sind zum Tranieren des Algorithmus nicht nur die Daten im gewählten Zeitrahmen genommen worden, sondern auch vorherliegende Datensätze. 
}
\subsection{Change Point Detection}
\subsection{Forecast}
\subsection{Userdefinierte Alerts}
\subsection{Justifications}
\section{Datenerfassung}
\section{Datenhaltung}
\section{Schnittstellen}
\section{Server}


\chapter{Ergebnis}
\section{Statistiken und Charts}
\section{Anomalien}
\section{Alerts}
\section{Ausblick}

\chapter{Resümee}


\chapter{Quellverzeichnis}

\chapter{Abbildungsverzeichnis}

\chapter{Glossar}

\chapter{Anhang}
\section{Architekturskizze}
\section{Datenmodell}
\section{Diplomarbeitsplakat}
\section{Abnahmeformular}

\end{document}
