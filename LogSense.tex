\documentclass{report}

\usepackage{titlesec}
\usepackage{enumitem}
\usepackage{xcolor}
\usepackage{microtype}

\definecolor{myblue}{RGB}{25,25,112}

\titleformat{\chapter}[display]
  {\normalfont\huge\bfseries\color{myblue}}{}{0pt}{\Huge}

\titleformat{\section}
  {\normalfont\Large\bfseries\color{myblue}}{}{0pt}{}

\titleformat{\subsection}
  {\normalfont\large\bfseries\color{myblue}}{}{0pt}{}

\titleformat{\subsubsection}
  {\normalfont\normalsize\bfseries\color{myblue}}{}{0pt}{}

\setlist[itemize]{label=$\bullet$, left=1em, labelsep=0.5em, itemsep=0.5em}
\setlist[enumerate]{label=\arabic*., left=1em, labelsep=0.5em, itemsep=0.5em}

\begin{document}

\maketitle

\tableofcontents

\chapter{Eidesstattliche Erklärung}
This chapter introduces the background and motivation behind the software engineering project.

\chapter{Gender-Erklärung}
Define the scope of the project, including any limitations and constraints.

\chapter{Danksagung}
Summarize relevant literature and related work in the field of software engineering.

\chapter{Impressum}
Clearly state the problem the project aims to address.

\chapter{Kurzfassung}
Define the specific objectives and goals of the software engineering project.
\textbf{Problemstellung} 

\chapter{Abstract}
Define the specific objectives and goals of the software engineering project.

\chapter{Inhaltsverzeichnis}

\chapter{Einleitung}
\section{Hintergrund}
\section{Zielsetzung}
\section{Projektinhalt}
\subsection{Projektumfeld}
\subsection{Projektteam}
\subsection{Betreuung}
\subsection{Auftraggeber}

\chapter{Theoretische und fachpraktische Grundlagen und Methoden}
\section{Verwendete Technologien}
\subsection{Java}
\subsection{Python}
\subsection{Postgres}
\subsection{nginx}
\subsection{Angular}
\section{Verwendete Entwicklungssysteme}
\subsection{PyCharm}
\subsection{IntelliJ}
\subsection{Webstorm}
\subsection{Datagrip}
\section{Verwendete Bibliotheken und Plug-Ins}
\subsection{Fast API}
\subsection{Pandas}
\subsection{Numpy}
\subsection{Scikit-Learn}
\subsection{Ruptures}
\subsection{Timescale DB}
\subsection{Charst JS}
\subsection{BootStrap}
\subsection{Angular Material}
\subsection{Oshi}

\section{Verwendete Architekturen}
\subsection{REST}
\subsection{Agent}
\section{Sonstige verwendete Software}
\subsection{GitLab}
\subsection{Adobe XD}
\subsection{LucidChart}

\chapter{Planung und Realisierung}
\section{Analyse und Entwurfsphase (Auszug aus Projektdokumente)}
\subsection{Funktionalität (UC, UC-Beschreibung)}
\subsection{Entwurf der Funktionalität}
\subsection{Datenmodell}

\chapter{Implementierung (Programmierung und QS/Test)}
\section{Agent}
\section{Server}
\section{DB}
\section{Rest API}
\section{Algorithmus}
\section{Frontend}

\chapter{Ergebnis}
\section{Statistiken und Charts}
\section{Anomalien}
\section{Alerts}


\chapter{Resümee}

\chapter{Aufgabenverteilung Verfassung Diplomarbeit (IMV)}
\section{Sahra Ettlinger}
\section{Thomas Jilek}
\section{Philipp Borbély}
\section{Emily Stadlbauer}

\chapter{Quellverzeichnis}

\chapter{Abbildungsverzeichnis}

\chapter{Glossar (Wortbeschreibungen)}

\chapter{Anhang}
\section{Diplomarbeitsplakat}
\section{Abnahmeformular}

\end{document}
