\documentclass[pdftex,11pt,a4paper]{book}
%\documentclass[dvips,11pt,a4paper,oneside]{book}

\usepackage {ifpdf}
\usepackage [american,german] {babel}
\usepackage[numbers,comma,square]{natbib}
\usepackage[utf8]{inputenc}
\usepackage{mya4wide}
\usepackage{fancyhdr}
\pagestyle{fancy}
\fancyhf{}
\rhead{LogSense}
\lhead{HTL Perg}
\renewcommand{\headrulewidth}{0pt}

\fancypagestyle{plain}{
  \fancyhf{}
  \rhead{LogSense}
  \lhead{HTL Perg}
  \fancyfoot[R]{\thepage} % Page number format on the right side
  \fancyfoot[L]{Borbely, Ettlinger, Jilek, Stadlbauer} % Names on the left side
  \renewcommand{\headrulewidth}{0pt}
}

\fancyfoot{}
\fancyfoot[R]{\thepage} % Page number format on the right side
\fancyfoot[L]{Borbely, Ettlinger, Jilek, Stadlbauer} % Names on the left side

\begin{document}

\maketitle

\tableofcontents

\chapter*{Eidesstattliche Erklärung}
This chapter introduces the background and motivation behind the software engineering project.

\chapter*{Gender-Erklärung}
Define the scope of the project, including any limitations and constraints.

\chapter*{Danksagung}
Summarize relevant literature and related work in the field of software engineering.

\chapter*{Impressum}
Clearly state the problem the project aims to address.

\chapter*{Kurzfassung}
Define the specific objectives and goals of the software engineering project.
\textbf{Problemstellung} 

\chapter*{Abstract}
Define the specific objectives and goals of the software engineering project.

\chapter*{Inhaltsverzeichnis}

\chapter{Einleitung}
\section{Ausgangssituation}
\section{Problemstellung}
\section{Zielsetzung}
\section{Projektinhalt}
\subsection{Anforderungen}
\subsection{Projektteam}
\subsubsection{Aufgabenverteilung}
\subsection{Auftraggeber}

\chapter{Theoretische und fachpraktische Grundlagen und Methoden}
\section{Visualiserung}
\subsection{Angular}
\subsection{Charst JS}
\subsection{BootStrap}
\subsection{Angular Material}

\section{Datenverarbeitung}
\subsection{Python}
\subsection{Pandas}
\subsection{Numpy}
\subsection{Scikit-Learn}
\subsection{Ruptures}
\subsection{Algorithmen}

\section{Datenerfassung}
\subsection{Agent}
\subsection{Java}
\subsection{Oshi}

\section{Datenhaltung}
\subsection{Postgres}
\subsection{Timescale DB}

\section{Schnittstellen}
\subsection{Fast API}
\subsection{REST}

\section{Server}
\subsection{Ubuntu}
\subsection{nginx}

\section{Entwicklungssysteme}
\subsection{PyCharm}
\subsection{IntelliJ}
\subsection{Webstorm}
\subsection{Datagrip}
\section{Sonstige Software}
\subsection{GitLab}
\subsection{Adobe XD}
\subsection{LucidChart}

\chapter{Planung und Realisierung}
\section{Funktionalität (UC, UC-Beschreibung)}
\section{Architektur}
\section{Informationsfluss}
\section{Datenmodell}

\chapter{Implementierung (Programmierung und QS/Test)}
\section{Visualiserung}
\section{Datenverarbeitung}
\subsection{Algorithmen}
\section{Datenerfassung}
\section{Datenhaltung}
\subsection{Datenbank}
\subsubsection{Einleitung}
\text{Die verwendete Datenbank ist TimescaleDB, welche auf Postgres aufbaut, der Vorteil dessen ist es, dass sowohl relationale Datenstrukturen, als auch die für Timeseries Data optimierten, sogenannten "Hypertable" zur verfügung stehen. Ein Vorteil dessen ist, dass TimescaleDB schnelle Lese -und Schreibgeschwindigkeiten für große Mengen an Timeseries-Daten anbietet, etwa wie die vom Agent gemessen Datensätze.} 

\title{Warum wird eine Datenbank benötigt?}
\text{Die vom Agent gemessen Daten müssen für spätere analyse und visualisierung persistiert werden. Weiters wird die wird die Datenbank zum abspeichern von Benutzerinformationen verwendet.}

\subsubsection{Datenmodell}
\subsubsection{Datenkatalog}
\text{Der Datenkatalog beschreibt die einzelnen Tabellen, deren Funktion und die jeweiligen Attribute.}
\subsection{Einfügen in Datenbank}
\section{Schnittstellen}
\section{Server}


\chapter{Ergebnis}
\section{Statistiken und Charts}
\section{Anomalien}
\section{Alerts}
\section{Ausblick}

\chapter{Resümee}


\chapter{Quellverzeichnis}

\chapter{Abbildungsverzeichnis}

\chapter{Glossar}

\chapter{Anhang}
\section{Architekturskizze}
\section{Datenmodell}
\section{Diplomarbeitsplakat}
\section{Abnahmeformular}

\end{document}
