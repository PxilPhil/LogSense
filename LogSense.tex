\documentclass[pdftex,11pt,a4paper]{book}
%\documentclass[dvips,11pt,a4paper,oneside]{book}

\usepackage {ifpdf}
\usepackage [american,german] {babel}
\usepackage[numbers,comma,square]{natbib}
\usepackage[utf8]{inputenc}
\usepackage{mya4wide}
\usepackage{fancyhdr}
\pagestyle{fancy}
\fancyhf{}
\rhead{LogSense}
\lhead{HTL Perg}
\renewcommand{\headrulewidth}{0pt}

\fancypagestyle{plain}{
  \fancyhf{}
  \rhead{LogSense}
  \lhead{HTL Perg}
  \fancyfoot[R]{\thepage} % Page number format on the right side
  \fancyfoot[L]{Borbely, Ettlinger, Jilek, Stadlbauer} % Names on the left side
  \renewcommand{\headrulewidth}{0pt}
}

\fancyfoot{}
\fancyfoot[R]{\thepage} % Page number format on the right side
\fancyfoot[L]{Borbely, Ettlinger, Jilek, Stadlbauer} % Names on the left side

\begin{document}

\maketitle

\tableofcontents

\chapter*{Eidesstattliche Erklärung}
This chapter introduces the background and motivation behind the software engineering project.

\chapter*{Gender-Erklärung}
Define the scope of the project, including any limitations and constraints.

\chapter*{Danksagung}
Summarize relevant literature and related work in the field of software engineering.

\chapter*{Impressum}
Clearly state the problem the project aims to address.

\chapter*{Kurzfassung}
Define the specific objectives and goals of the software engineering project.
\textbf{Problemstellung} 

\chapter*{Abstract}
Define the specific objectives and goals of the software engineering project.

\chapter*{Inhaltsverzeichnis}

\chapter{Einleitung}
\section{Ausgangssituation}
\section{Problemstellung}
\section{Zielsetzung}
\section{Projektinhalt}
\subsection{Anforderungen}
\subsection{Projektteam}
\subsubsection{Aufgabenverteilung}
\subsection{Auftraggeber}

\chapter{Theoretische und fachpraktische Grundlagen und Methoden}
\section{Visualiserung}
\subsection{Angular}
\subsection{Charst JS}
\subsection{BootStrap}
\subsection{Angular Material}
\section{Datenverarbeitung}
\subsection{Python}
\subsection{Pandas}
\text{Pandas ist eine Python-Bibliothek zur Datenverarbeitung und -analyse. Sie stellt dabei verschiedene Datenstrukturen und Funktionen für das Manipulieren von tabellarischen und strukturierten Daten zur Verfügung. Zu den wichtigsten Strukturen gehören DataFrames und Series.  Series sind eindimensionale, array-ähnliche Objekte welche ähnlich einer einzelnen Spalte in einer Tabelle sind. Bei DataFrames handelt es sich um zweidimensionale Tabellen mit Zeilen und Spalten. Einzelne Spalten von einem DataFrame können dabei verschiedene Datentypen aufweisen. Zu den wichtigsten Funktionen gehören Aggregationen, Gruppierungen und Statistische Operationen. Bei dieser Diplomarbeit wird Pandas bei der Datenverarbeitung eingesetzt, insbesondere in Kombination mit Scikit-Learn, Numpy und Ruptures zur Identifikation von Anomalien, Change-Points und zur Bereining und Zusammenführung von Datensätzen.}
\subsection{Numpy}
\subsection{Scikit-Learn}
\subsection{Ruptures}
\subsection{Algorithmen}
\section{Datenerfassung}
\subsection{Agent}
\subsection{Java}
Die Programmiersprache Java ist objektorientiert, besitzt eine einfache Syntax mit überschaubarem Sprachumfang und ist durch kontrollierte Speicherzugriffe sicher. Wenn es eine passende JVM (= Java Virtual Machine) für das Betriebssystem gibt, kann das Programm auf diesem Betriebssytem laufen. Diese Plattformunabhängig wird dadurch ermöglicht, dass der Java-Quellcode in Bytecode übersetzt wird, der dann von der JVM interpretiert werden kann.\\
Der Agent, der lokal auf den Windows Rechnern läuft und die Daten über den Ressourcenverbrauch des Rechners erfasst, ist in der Programmiersprache Java entwickelt. Der Grund dafür ist, dass es für Java viele verschiedene Bibliotheken gibt, die auf die Hardwaretreiber zugreifen, um die benötigten Daten über den Rechner, die darauf laufenden Programme und deren Ressourcenverbrauch auszulesen.
\subsection{Oshi}
\section{Datenhaltung}
\subsection{Postgres}
\subsection{Timescale DB}
\section{Schnittstellen}
\subsection{Fast API}
\subsection{REST}
\section{Server}
\subsection{Ubuntu}
\subsection{nginx}

\section{Entwicklungssysteme}
\subsection{PyCharm}
\subsection{IntelliJ}
\subsection{Webstorm}
\subsection{Datagrip}
\section{Sonstige Software}
\subsection{GitLab}
\subsection{Adobe XD}
\subsection{LucidChart}

\chapter{Planung und Realisierung}
\section{Funktionalität (UC, UC-Beschreibung)}
\section{Architektur}
\section{Informationsfluss}
\section{Datenmodell}

\chapter{Implementierung (Programmierung und QS/Test)}
\section{Visualiserung}
\section{Datenverarbeitung}
\subsection{Algorithmen}
\section{Datenerfassung}
\section{Datenhaltung}
\section{Schnittstellen}
\section{Server}


\chapter{Ergebnis}
\section{Statistiken und Charts}
\section{Anomalien}
\section{Alerts}
\section{Ausblick}

\chapter{Resümee}


\chapter{Quellverzeichnis}

\chapter{Abbildungsverzeichnis}

\chapter{Glossar}

\chapter{Anhang}
\section{Architekturskizze}
\section{Datenmodell}
\section{Diplomarbeitsplakat}
\section{Abnahmeformular}

\end{document}
