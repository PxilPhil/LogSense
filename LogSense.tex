\documentclass[pdftex,11pt,a4paper]{book}
%\documentclass[dvips,11pt,a4paper,oneside]{book}

\usepackage {ifpdf}
\usepackage [american,german] {babel}
\usepackage[numbers,comma,square]{natbib}
\usepackage[utf8]{inputenc}
\usepackage{mya4wide}
\usepackage{fancyhdr}
\pagestyle{fancy}
\fancyhf{}
\rhead{LogSense}
\lhead{HTL Perg}
\renewcommand{\headrulewidth}{0pt}

\fancypagestyle{plain}{
  \fancyhf{}
  \rhead{LogSense}
  \lhead{HTL Perg}
  \fancyfoot[R]{\thepage} % Page number format on the right side
  \fancyfoot[L]{Borbely, Ettlinger, Jilek, Stadlbauer} % Names on the left side
  \renewcommand{\headrulewidth}{0pt}
}

\fancyfoot{}
\fancyfoot[R]{\thepage} % Page number format on the right side
\fancyfoot[L]{Borbely, Ettlinger, Jilek, Stadlbauer} % Names on the left side

\begin{document}

\maketitle

\tableofcontents

\chapter*{Eidesstattliche Erklärung}
This chapter introduces the background and motivation behind the software engineering project.

\chapter*{Gender-Erklärung}
Define the scope of the project, including any limitations and constraints.

\chapter*{Danksagung}
Summarize relevant literature and related work in the field of software engineering.

\chapter*{Impressum}
Clearly state the problem the project aims to address.

\chapter*{Kurzfassung}
Define the specific objectives and goals of the software engineering project.
\textbf{Problemstellung} 

\chapter*{Abstract}
Define the specific objectives and goals of the software engineering project.

\chapter*{Inhaltsverzeichnis}

\chapter{Einleitung}
\section{Ausgangssituation}
\section{Problemstellung}
\section{Zielsetzung}
\section{Projektinhalt}
\subsection{Anforderungen}
\subsection{Projektteam}
\subsubsection{Aufgabenverteilung}
\subsection{Auftraggeber}

\chapter{Theoretische und fachpraktische Grundlagen und Methoden}
\section{Visualiserung}
\subsection{Angular}
\subsection{Charst JS}

\text
Chart.js ist eine JavaScript Bibliothek mit deren Hilfe man unterschiedliche Diagrammtypen in Webanwendungen darstellen kann. 
Es können 8 unterschiedliche Diagrammarten dargestellt werden, darunter: 
\begin{itemize}
    \item Liniendiagramme
    \item Balkendiagramme
    \item Bereichsdiagramme
    \item Kreisdiagramme
    \item Tortendiagramme
    \item Radardiagramme
    \item Polar Diagramme
    \item Streudiagramme
\end{itemize}
Die Bibliothek bietet unterschiedlieche, interaktive Funktionen wie Tooltipps und Animationen. 
Darüber hinaus gibt es viele Möglichkeiten mit verschiedenen Optionen und Konfigurationen das Diagramm nach eigenen Wünschen  anzupassen. 
Chart.js unterstützt responsives Design und kann sich an unterschiedliche Bildschirmgrößen anpassen.\\In dieser Diplomarbeit ist Chart.js im Angular Projekt eingebunden um unterschiedliche Messwerte grafisch darzustellen.


\subsection{BootStrap}
\subsection{Angular Material}
\section{Datenverarbeitung}
\subsection{Python}
\subsection{Pandas}
\subsection{Numpy}
\subsection{Scikit-Learn}
\subsection{Ruptures}
\subsection{Algorithmen}
\section{Datenerfassung}
\subsection{Agent}
\subsection{Java}
\subsection{Oshi}
\section{Datenhaltung}
\subsection{Postgres}
\subsection{Timescale DB}
\section{Schnittstellen}
\subsection{Fast API}
\subsection{REST}
\section{Server}
\subsection{Ubuntu}
\subsection{nginx}

\section{Entwicklungssysteme}
\subsection{PyCharm}
\subsection{IntelliJ}
\subsection{Webstorm}
\subsection{Datagrip}
\section{Sonstige Software}
\subsection{GitLab}
\subsection{Adobe XD}
\subsection{LucidChart}

\chapter{Planung und Realisierung}
\section{Funktionalität (UC, UC-Beschreibung)}
\section{Architektur}
\section{Informationsfluss}
\section{Datenmodell}

\chapter{Implementierung (Programmierung und QS/Test)}
\section{Visualiserung}
\section{Datenverarbeitung}
\subsection{Algorithmen}
\section{Datenerfassung}
\section{Datenhaltung}
\section{Schnittstellen}
\section{Server}


\chapter{Ergebnis}
\section{Statistiken und Charts}
\section{Anomalien}
\section{Alerts}
\section{Ausblick}

\chapter{Resümee}


\chapter{Quellverzeichnis}

\chapter{Abbildungsverzeichnis}

\chapter{Glossar}

\chapter{Anhang}
\section{Architekturskizze}
\section{Datenmodell}
\section{Diplomarbeitsplakat}
\section{Abnahmeformular}

\end{document}
